\documentclass[11pt]{m2pi}
%%%%%%%%%%%%%%%%%%%%%%%%%%%%%%%%%%%%%%%%%%%%%%%%%%%%%%%%%%%%%%%%%%%%%%
%                          STANDARD PACKAGES                         %
%%%%%%%%%%%%%%%%%%%%%%%%%%%%%%%%%%%%%%%%%%%%%%%%%%%%%%%%%%%%%%%%%%%%%%

\usepackage{graphicx}
\graphicspath{{figures/}}

\usepackage{epstopdf}
\usepackage{amsthm}
\usepackage{amsmath}
\usepackage{amssymb}
\usepackage{amsxtra}
\usepackage{amscd}
\usepackage{verbatim}
\usepackage{rotating}
\usepackage{multirow}
\usepackage{multicol}
\usepackage{url}
\usepackage[margin=1.5in]{geometry}
\usepackage[all]{xy}

%%%%%%%%%%%%%%%%%%%%%%%%%%%%%%%%%%%%%%%%%%%%%%%%%%%%%%%%%%%%%%%%%%%%%%
%                        STANDARD ENVIRONMENTS                       %
%%%%%%%%%%%%%%%%%%%%%%%%%%%%%%%%%%%%%%%%%%%%%%%%%%%%%%%%%%%%%%%%%%%%%%

% Theorems

%Commented out this theorem style I  have no idea why it's even here - Erik
%\theoremstyle{theorem}
\newtheorem{theorem}{Theorem}[section]
\newtheorem{alphtheorem}{Theorem}
\renewcommand{\thealphtheorem}{\Alph{alphtheorem}}
\newtheorem{lemma}[theorem]{Lemma}
\newtheorem{corollary}[theorem]{Corollary}
\newtheorem{guess}[theorem]{Conjecture}
\newtheorem{definition}[theorem]{Definition}
\newtheorem{facts}[theorem]{Facts}
\newtheorem{proposition}[theorem]{Proposition}
\newtheorem{problem}[theorem]{Problem}
%\newtheorem{algorithm}[theorem]{Algorithm}
%\newtheorem{claim}{Claim}[theorem]
\newtheorem{claim}[theorem]{Claim}
\newtheorem{subclaim}{Claim}[theorem]
\renewcommand{\thesubclaim}{\thetheorem.\alph{subclaim}}
\newtheorem{example}[theorem]{Example}
%\newtheorem*{example*}{Example}
\newtheorem{result}[theorem]{Result}
\newtheorem{observation}[theorem]{Observation}
\newtheorem{numconjecture}[theorem]{Conjecture}

\newtheoremstyle{myexample}{3pt}{3pt}{}{}{\bfseries}{.}{ }{\thmname{#1}\thmnumber{ #2}\thmnote{ (#3)}}
\theoremstyle{myexample}
\newtheorem{numremark}[theorem]{Remark}

\newtheoremstyle{myremark}{3pt}{3pt}{}{}{\bfseries}{.}{ }{\thmname{#1}\thmnote{ (#3)}}
\theoremstyle{myremark}
\newtheorem{remark}{Remark}
\newtheorem*{remark*}{Remark}
\newtheorem*{remarks*}{Remarks}
\newtheorem*{observation*}{Observation}
\newtheorem*{example*}{Example}

\newtheoremstyle{conjecture}{3pt}{3pt}{\itshape}{}{\bfseries}{.}{ }{\thmname{#1}\thmnote{ (#3)}}
%\newtheoremstyle{conjecture}{3pt}{3pt}{\rmfamily}{}{\itshape}{.}{ }{\thmname{#1}}
\theoremstyle{conjecture}
\newtheorem{question}{Question}
\newtheorem*{question*}{Question}
\newtheorem{conjecture}{Conjecture}
\newtheorem{theorem*}{Theorem}


\numberwithin{equation}{section}

% References
\newcommand{\genericref}[2]{#1~\ref{#2}}
\newcommand{\generictworef}[3]{#1~\ref{#2} and \ref{#3}}
\newcommand{\genericthreeref}[4]{#1~\ref{#2}, \ref{#3} and \ref{#4}}
\newcommand{\thmref}[1]{\genericref{Theorem}{#1}}
\newcommand{\lemmaref}[1]{\genericref{Lemma}{#1}}
\newcommand{\claimref}[1]{\genericref{Claim}{#1}}
\newcommand{\remarkref}[1]{\genericref{Remark}{#1}}
\newcommand{\conjref}[1]{\genericref{Conjecture}{#1}}
\newcommand{\tableref}[1]{\genericref{Table}{#1}}
\newcommand{\figref}[1]{\genericref{Figure}{#1}}
\newcommand{\twofigref}[2]{\generictworef{Figures}{#1}{#2}}
\newcommand{\threefigref}[3]{\genericthreeref{Figures}{#1}{#2}{#3}}
\newcommand{\propref}[1]{\genericref{Proposition}{#1}}
\newcommand{\twopropref}[2]{\generictworef{Propositions}{#1}{#2}}

% Case environment
\newenvironment{genericcase}[6]{\medskip\par\noindent \csname#1\endcsname{#2 #3}#4 \csname#5\endcsname{#6}.\begin{indentation}{1.5em}{0em}\noindent\ignorespaces}{\end{indentation}}
\newenvironment{case}[2]{\noindent \textit{Case #1}: #2.\begin{indentation}{1.5em}{0em}}{\end{indentation}}

%Note that I changed  \textit{Subcase #1}. #2. to  \textit{Subcase #1}: #2.
%Erik

\newenvironment{subcase}[2]{\noindent \textit{Subcase #1}: #2.\begin{indentation}{1.5em}{0em}}{\end{indentation}}
\newenvironment{ulindent}[1]{\noindent \underline{#1:}\begin{indentation}{1.5em}{0em}\setlength{\parindent}{0em}}{\end{indentation}}
\newenvironment{bfcase}[1]{\par\medskip\begin{indentation}{1.5em}{0em}\noindent\hspace*{-1.5em}\textbf{#1:}}{\end{indentation}}
\newenvironment{itcase}[1]{\par\medskip\begin{indentation}{1.5em}{0em}\noindent\hspace*{-1.5em}\textit{#1:}}{\end{indentation}}
%\newcommand{\case}[2]{\noindent \textbf{Case #1}: #2:}

% Induction environment
\newenvironment{basecase}[1]{\noindent \textsc{Base case}: #1:\begin{indentation}{1.5em}{0em}}{\end{indentation}}
\newenvironment{inductionstep}[1]{\noindent \textsc{Induction step}: #1:\begin{indentation}{1.5em}{0em}}{\end{indentation}}

% Algorithm headers
\newcounter{algorithm}
\setcounter{algorithm}{0}
\renewcommand{\thealgorithm}{\thesection.\arabic{algorithm}}
\newenvironment{algorithm}[1]{%
  \null
  \refstepcounter{algorithm}%
  \hrule%
  \vspace{0.2em}%
  \noindent\textbf{Algorithm \thealgorithm} #1
  \vspace{0.2em}%
  \hrule%
  \vspace{0.2em}
  }{%
  \vspace{0.2em}%
  \hrule%
  \null
}

%%%%%%%%%%%%%%%%%%%%%%%%%%%%%%%%%%%%%%%%%%%%%%%%%%%%%%%%%%%%%%%%%%%%%%
%                          STANDARD COMMANDS                         %
%%%%%%%%%%%%%%%%%%%%%%%%%%%%%%%%%%%%%%%%%%%%%%%%%%%%%%%%%%%%%%%%%%%%%%

\newcommand{\ie}{i.e.}
\newcommand{\eg}{e.g.} 
\newcommand{\resp}{\emph{resp.\@\ }} 
\newcommand{\etal}{et~al.} 
\newcommand{\cf}{cf.}

\newcommand{\mathbif}[1]{\mathbf{\emph{#1}}}
\newcommand{\nth}[1]{\ensuremath{#1^{\textrm{th}}}}
\newcommand{\nrd}[1]{\ensuremath{#1^{\textrm{rd}}}}
\newcommand{\nst}[1]{\ensuremath{#1^{\textrm{st}}}}
\newcommand{\jth}{\nth{j}}
\newcommand{\kth}{\nth{k}}
\newcommand{\qth}{\nth{q}}
\newcommand{\rth}{\nth{r}}

\newcommand{\vfrac}[2]{\ensuremath{{}^{\displaystyle#1} \diagup {}_{\displaystyle#2}}}

\renewcommand{\pmod}[1]{\ \left({\rm mod\ } #1 \right)}

%%%%%%%%%%%%%%%%%%%%%%%%%%%%%%%%%%%%%%%%%%%%%%%%%%%%%%%%%%%%%%%%%%%%%%
%                      SAVING THEOREM NUMBERS                        %
%%%%%%%%%%%%%%%%%%%%%%%%%%%%%%%%%%%%%%%%%%%%%%%%%%%%%%%%%%%%%%%%%%%%%%

%\newcounter{temp}
%\def\savecounter#1{\newcounter{#1}\setcounter{#1}{\value{theorem}}}
%\def\changetheoremcounter#1{\setcounter{temp}{\value{theorem}}\setcounter{theorem}{\value{#1}}}%\csname the#1\endcsname}}
%\def\restoretheoremcounter{\setcounter{theorem}{\value{temp}}}

\newcounter{temp}
\newcounter{ctemp}
\def\savecounter#1#2{%
   \newcounter{#1}%
   \setcounter{#1}{\value{section}}%
   \newcounter{#2}\setcounter{#2}{\value{theorem}}}
\def\swapcounter#1#2{%
   \setcounter{ctemp}{\value{section}}%
   \setcounter{section}{\value{#1}}%
   \setcounter{temp}{\value{theorem}}%
   \setcounter{theorem}{\value{#2}}%
   \addtocounter{theorem}{-1}}
\def\restorecounter{%
   \setcounter{section}{\thectemp}%
   \setcounter{theorem}{\thetemp}}


% Insert custom commands and necessary packages here.

\begin{document}

% A short title is not required, but if needed use:
% \title[short title]{full title}
\title[Short title]{Title}

 \author{Author}
\address{}
\email{}

% For each additional author, add another set of
% \author, \address, and \email commands


% \thanks entries are to acknowledge grants. You may combine
% all acknowledgments into one \thanks entry, or may use
% multiple \thanks entries. They generate footnotes without
% tags, so you must be explicit about which authors are
% thanking whom.
\thanks{}

\begin{abstract}
 Let $\varepsilon'$ be a multiply null homomorphism.  The authors address the invertibility of rings under the additional assumption that $\chi$ is larger than $\tilde{r}$.  Recently, there has been much interest in the construction of planes. In this setting, the ability to describe everywhere contravariant domains is essential.
\end{abstract}

\maketitle
\section{Here is a section.}
Lorem ipsum dolor sit amet, consectetur adipiscing elit, sed do eiusmod tempor incididunt ut labore et dolore magna aliqua. Ut enim ad minim veniam, quis nostrud exercitation ullamco laboris nisi ut aliquip ex ea commodo consequat. 
\[\int\int e^{-x^2 + y^2} \operatorname{d}\!y \operatorname{d}\!x\]
Duis aute irure dolor in reprehenderit in voluptate velit esse cillum dolore eu fugiat nulla pariatur. Excepteur sint occaecat cupidatat non proident, sunt in culpa qui officia deserunt mollit anim id est laborum.
	

\begin{theorem}
For all $n\in \mathbb{Z}^{+}$ odd, $\zeta(n) = \pi^{2}/6.$
\end{theorem}

\begin{proof}
The proof is trivial and left as an exercise for the reader.
\end{proof}

\begin{definition}
We call a graph $G$ \emph{weakly bipartite} if for all $\epsilon < 0$, there exists a $\delta < 0$ such that $G \sim E$. 
\end{definition}

\begin{corollary}
Let us assume we are given an anti-multiplicative, completely continuous equation $G$.  Then $\emptyset \to \sin^{-1} \left( {\mathfrak{{v}}_{z}}^{-7} \right)$.
\end{corollary}

\begin{lemma}
Suppose $U$ is controlled by $e$.  Then $$\overline{{\mathcal{{A}}_{Z,O}}} = \bigotimes_{\bar{X} = 0}^{\sqrt{2}}  \int \exp^{-1} \left( \aleph_0^{-1} \right) \,d u.$$
\end{lemma}

\begin{example}
Let $\varphi \supset \Omega$. By the uniqueness of Bernoulli topoi, if the Riemann hypothesis holds then $0^{8} = \sin^{-1} \left( T \| \omega \| \right)$. By the surjectivity of standard, local subalgebras, if $R''$ is not isomorphic to $\Phi'$ then $M \ne 0$. On the other hand, if $\| \psi \| \in {\gamma^{(v)}}$ then every scalar is standard and co-abelian. Clearly, there exists a countable, complete and dependent co-regular number acting canonically on an Eratosthenes, quasi-smoothly isometric, semi-elliptic number. Note that there exists a compactly Atiyah--Riemann, Green--Leibniz, hyperbolic and left-meromorphic natural isomorphism.
 The result now follows by an approximation argument.
\end{example}

\begin{proposition}
Suppose we are given an everywhere uncountable point $\mathfrak{{f}}$.  Let ${\mathfrak{{w}}_{\mathcal{{J}}}} <-\infty$ be arbitrary.  Further, let $\mathbf{{a}} \equiv 1$ be arbitrary.  Then \begin{align*} \aleph_0 {k_{\mathfrak{{p}},\sigma}} & = \omega \left(-\mathfrak{{b}}'' \right)-\zeta \left( \frac{1}{t}, \dots, \xi \right) + \dots \cup \tanh^{-1} \left( \frac{1}{0} \right)  \\ & < \varinjlim_{\delta' \to 1}  \overline{e^{-3}} \cup \dots-\mathcal{{W}}^{-1} \left(-\hat{I} \right)  \\ & < \iint_{0}^{2} L \left(-1 1, \dots, \emptyset-\infty \right) \,d \mathcal{{L}} \pm \sinh \left( 0^{3} \right) \\ & \subset \left\{ \frac{1}{\sqrt{2}} \colon \log^{-1} \left(-\hat{\mathfrak{{m}}} \right) = \frac{-1 + {\mathfrak{{\ell}}_{C,R}}}{\overline{\frac{1}{\emptyset}}} \right\} .\end{align*}
\end{proposition}

\section{Conclusion}

The authors address the existence of points under the additional assumption that $\Omega \ne i$. In this setting, the ability to examine ideals is essential. A central problem in modern operator theory is the derivation of globally Bernoulli subrings. This could shed important light on a conjecture of Deligne. It was Shannon who first asked whether differentiable, universally open, hyper-free topoi can be constructed. Hence in \cite{cite:24}, it is shown that ${z_{L}}$ is $I$-almost integral.


% Thank people here. Replace by the plural ``Acknowledgements'' if thanking multiple people.
\section*{Acknowledgement}
The author's wish to thank Willy Wonka for many helpful discussions which contributed to the quality of this paper.

\nocite{*}
\bibliographystyle{amsplain}
\bibliography{}


\end{document}
